\section*{\fontsize{16}{14}\selectfont Aim: Conduct feasibility study for some given problem.}
\begin{center}\underline{\textbf{Project: Certificate Generation System}}\end{center}
Certifiacte Generation System(CGS) application is a portable application used to Generate Certificate for single candidate providing his/her details along with image,
As well as for Batch/Number of candidates by simply providing the CSV format file (containing details of every candidate) along with candidate images in a compressed (tar.gz or zip) folder.\\\\
This project can be used in various Universities, Organisations and other sectors for honouring individuals.\\\\
This can generate the end output (i.e. Certificate) in various graphic forms. This will also contain candidate's image.\\\\ 
Also, this project is completely open source and the entire code is available 
to the user as and when required. There is Complete developer's 
Documentation as well as User manual alongwith it that helps using it a lot easier.

\begin{center}\underline{\textbf{Feasibility Study}}\end{center}
Feasibility study is made to see if the project on completion will serve the purpose of the organization for the amount of work, effort and the time that spend on it. Feasibility study lets the developer foresee the future of the project and the usefulness.\\\\
Objectives of feasibility study are listed below:
\begin{itemize}
	\item To analyze whether the software will meet organizational requirements.
	\item To determine whether the software can be implemented using the current technology and within the specified budget and schedule.
	\item To determine whether the software can be integrated with other existing software.
\end{itemize}
\textbf{\emph{An Overview of the Study}}\\\\
\textbf{Feasibility study involves:}
\begin{enumerate}
	\item \textbf{Technical Feasibility:} Technical feasibility is carried out to determine whether the project has the capability, in terms of software and hardware to handle and fulfill the user requirements. The assessment is based on an outline design of system requirements in terms of Input, Processes, Output, Cross-Platform and Procedures.\\
Technological issues raised during the investigation are:
\begin{itemize}
	\item Does the existing technology sufficient for the suggested one?
	\item Can the system expand if developed?
\end{itemize}
  \emph{Languages/Technologies used:}
\begin{itemize}
\item HTML \& CSS
\item Libreoffice Language
\item PHP
\item Apache(Web Server)
  \end{itemize}      
	\item \textbf{Economical Feasibility:} It includes quantification and identification of all the benefits expected. This assessment typically involves a cost/ benefits analysis. It is important to identify cost and benefit factors, which can be categorized as follows:
\begin{itemize}
\item  \emph{Development \& Operation Cost:} It's an open source project and hence free of cost for users.
\end{itemize}
	\item \textbf{Market Feasibility:} The market needs analysis to view the market demand,competitive activities etc. There is a need for this project for following things:
\begin{itemize}
\item  Generate large number of Certificates at a time.
\item  Make it work like batch mode. So, that user can give inputs together and relax.
\item  User Interactive Certificates.
\item Reduce the time for analysis.
\item Provide on-line way to analysis so that individual does not have to install anything.
\end{itemize}
\item \textbf{Legal Feasibility:} It investigates if the proposed system conflicts with legal requirements like data protection acts or social media laws.
\item \textbf{Safety Feasibility:} It refers to an analysis of whether a project is capable of being implemented and operated safely with minimum adverse effects on the environment.
        \item \textbf{Behavioral Feasibility:} Behavioral feasibility assesses the extent to which the required software performs a series of steps to solve business problems and user requirements. It is a measure of how well the solution of problems or a specific alternative solution will work in the organization.\\\\
\emph{Functions:}\\
The user has been provided some test functions which he can use to test various.\\\\
\emph{Plots:}\\
CGS provides plotting of image. The size of the image should be less than 400kb.\\\\
\emph{Input:}\\
Input values are taken from user or default values defined in the file are used.\\\\
\emph{Output:}\\
The iterations are performed and it returns the output with the expected precision.
\item \textbf{Operational Feasibility:} Operational feasibility is a measure of how well a project solves the problems, and takes advantage of the opportunities identified during scope definition and how it satisfies the requirements analysis phase of system  development.All the operations performed in the software are very quick and satisfy all the requirements.\\\\
\emph{Hardware Requirements:}
\begin{itemize}
\item Operating System: Linux/Windows
\item Processor Speed: 512KHz or more
\item RAM: Minimum 256MB
\end{itemize}
         \emph{Software Requirements:}
\begin{itemize}
\item Software: Libreoffice
\item Programming Language: Php, Html/CSS, Libreoffice
\end{itemize}
\end{enumerate}
