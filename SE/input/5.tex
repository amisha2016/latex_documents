\section*{\fontsize{16}{14}\selectfont Aim: Preparation of Software Configuration Management and Risk Management related documents}
}
The purpose of Software Configuration Management is to establish and maintain the integrity
of the products of the software project throughout the project's software life cycle. Software
Configuration Management involves identifying configuration items for the software project,
controlling these configuration items and changes to them, and recording and reporting status
and change activity for these configuration items.
 
Configuration management (CM) refers to a discipline for evaluating, coordinating,
approving or disapproving, and implementing changes in artefacts that are used to construct
and maintain software systems. An artifact may be a piece of hardware or software or
documentation. CM enables the management of artifact from the initial concept through
design, implementation, testing, base lining, building, release, and maintenance.

CM is intended to eliminate the confusion and error brought about by the existence of
different versions of artifacts. Artifact change is a fact of life: plan for it or plan to be overwhelmed by it. Changes are made to correct errors, provide enhancements, or simply reflect the evolutionary refinement of product definition. CM is about keeping the inevitable
change under control. Without a well-enforced CM process, different team members (possibly at different sites) can use different versions of artifacts unintentionally. Successful CM requires a well-defined and institutionalized set of policies and standards that clearly define
\begin{itemize}

\item The set of artifacts (configuration items) under the jurisdiction of CM (Models that
will be used)
\item
How artifacts are named (division of models into admin, student and company

module)
\item How artifacts enter and leave the controlled set (various logins and logouts)
\item How an artifact under CM is allowed to change (admin has the sole authority
)\item
How different versions of an artifact under CM are made available and under what
conditions each one can be used (whether to register , participate or work as admin)
\item
How CM tools are used to enable and enforce CM
\end{itemize}
\begin{itemize}

\item In single-system CM, each version of the system has a configuration associated with it that
defines the versions of the configuration items that went into that system's production. In
product line CM, a configuration must be maintained for each version of each product.
\item In single-system CM, each product with all of its versions maybe managed separately. In product line CM, such management is untenable, because the core assets are used across all products. Hence, the entire product line is usually managed with a single, unified CM process.
\item Product line CM must control the configuration of the core asset base and its use by all product developers. It must account for the fact that core assets are usually produced by one team and used in parallel by several others. Single-system CM has no such burden:
the component developers and product developers are the same people.
\item Only the most capable CM tools can be used in a product line effort. Many tools that are adequate for single-system CM are simply not sufficiently robust to handle the demands of product line CM. (See the "Tool Support" practice area for a more complete discussion of tools.)
\end{itemize}

The mission of product line CM is allowing the rapid reconstruction of any product version that may have been built using various versions of the core assets and development/operating environment plus various versions of product-specific artifacts. One product line manager summed up the problem.

\subsection*{Risk Management Plan:}
\subsubsection*{Purpose of the Risk management plan}

A risk is an event or condition that, if it occurs, could have a positive or negative effect on a
projects objectives. Risk Management is the process of identifying, assessing, responding to,
monitoring, and reporting risks. This Risk Management Plan defines how risks associated
with the <Project Name> project will be identified, analyzed, and managed. It outlines how
risk management activities will be performed, recorded, and monitored throughout thelifecycle of the project and provides templates and practices for recording and prioritizing risks.

The Risk Management Plan is created by the project manager in the Planning Phase of the CDC Unified Process and is monitored and updated throughout the project. The intended audience of this document is the project team, project sponsor and management.
\subsubsection*{Risk management Procedure:}
The project manager working with the project team and project sponsors will ensure that risks are actively identified, analyzed, and managed throughout the life of the project. Risks will be identified as early as possible in the project so as to minimize their impact. The steps foraccomplishing this are outlined in the following sections. The <project manager or other designee> will serve as the Risk Manager for this project.
\subsubsection*{Risk Identification}
Risk identification will involve the project team, appropriate stakeholders, and will include an
evaluation of environmental factors, organizational culture and the project management plan
including the project scope. Careful attention will be given to the project deliverables,
assumptions, constraints, WBS, cost/effort estimates, resource plan, and other key project documents.

A Risk Management Log will be generated and updated as needed and will be stored
electronically in the project library located at <file location>.
In our project Function Organiser also we looked at various aspects of risks that could occur.
We looked at the environmental factors and our organizational culture pertained to the student body of the college Project scope included serving and notifications of various events for which the students registered, so that it reduces the manual work and the system of organizing events could come easy. Assumptions of student body and companies that would participate and cost of project was also designed.
\subsubsection*{Risk Analysis}
All risks identified will be assessed to identify the range of possible project outcomes. Qualification will be used to determine which risks are the top risks to pursue and respond to and which risks can be ignored Qualitative Risk Analysis.
The probability and impact of occurrence for each identified risk will be assessed by the project manager, with input from the project team using the following approach:
\begin{description}
\item{Probability}
\begin{itemize}
\item \textbf{High} Greater than 70 probability of occurrence
\item\textbf{ Medium} Between 30 and 70 probability of occurrence
\item \textbf{Low} Below 30 probability of occurrence
\end{itemize}

\item{Impact}

\begin{itemize}

\item \textbf{High} Risk that has the potential to greatly impact project cost, project schedule or performance (Security for Functions defination)
\item\textbf{ Medium } Risk that has the potential to slightly impact project cost, project schedule or performance (Students not aware of the functionality of the project Function Organiser.)
\item \textbf{Low} Risk that has relatively little impact on cost, schedule or Performance (connection or connectivity issues as in terms of the project)

 \end{itemize}
 
 \end{description}
Risks that fall within the RED and YELLOW zones will have risk response planning which
may include both a risk mitigation and a risk contingency plan.
\subsubsection*{Quantitative Risk Analysis}
Analysis of risk events that have been prioritized using the qualitative risk analysis process and their effect on project activities will be estimated, a numerical rating applied to each risk based on this analysis, and then documented in this section of the risk management plan.

