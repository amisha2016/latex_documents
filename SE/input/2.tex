\section*{\fontsize{16}{14}\selectfont Aim: Preparation of Software Requirement Specification Document.}
An SRS is basically
 an organization's understanding (in writing) of a customer or potential
client's sy
stem requirements and dependencies at a particular point in time (usually) prior to
any
 actual design or development work. It's a two-way
 insurance policy
 that assures that both
the client and the organization understand the other's requirements from that perspective at a
given point in time.\\

The SRS document itself states in precise and explicit language those functions and
capabilities a software sy
stem (i.e., a software application, an ecommerce Web site, and so
on) must provide, as well as states any
 required constraints by
 which the sy
stem must abide.\\

The SRS also functions as a blueprint for completing a project with as little cost growth as
possible. The SRS is often referred to as the "parent" document because all subsequent
project management documents, such as design specifications, statements of work, software
architecture specifications, testing and validation plans, and documentation plans, are related
to it.\\

It's important to note that an SRS contains functional and nonfunctional requirements only
; it
doesn't offer design suggestions, possible solutions to technology
 or business issues, or any other information other than what the development team understands the customer's system
requirements to be.
A well-designed, well-written SRS accomplishes four major goals:
\begin{itemize}
\item
It provides feedback to the customer. The SRS should be
written in natural language (versus a formal language, explained later in this article),
in an unambiguous manner that may
 also include charts, tables, data flow diagrams,
decision tables, and so on.

\item
It decomposes the problem into component parts. The simple act of writing down
software requirements in a well-designed format organizes information, placesborders around the problem, solidifies ideas, and helps break down the problem into
its component parts in an orderly
 fashion.

\item
The SRS serves as the parent document to subsequent documents, such as the software design specification and statement of work. Therefore, the SRS must contain sufficient detail in the functional system requirements so that a design solution can be devised.

\item It serves as a product validation check. The SRS also serves as the parent document
for testing and validation strategies that will be applied to the requirements for
verification.

\end{itemize}
\textbf{What Kind of Information Should an SRS Include?}\\

You probably
 will be a member of the SRS team (if not, ask to be), which means SRS
development will be a collaborative effort for a particular project. In these cases, y
our
company
 will have developed SRSs before, so y
ou should have examples (and, likely
, the
company
's SRS template) to use. But, let's assume y
ou'll be starting from scratch. Several
standards organizations (including the IEEE) have identified nine topics that must be
addressed when designing and writing an SRS:

\begin{enumerate}

\item Interfaces
\item Functional Capabilities
\item Performance Levels
\item Data Structures
/Elements
\item Safety

\item Reliability

\item Security
/Privacy

\item Quality

\item Constraints and Limitations

\end{enumerate}
\textbf{Begin with an SRS Template}\\

The first and biggest step to writing an SRS is to select an existing template that y
ou can fine
tune for y
our organizational needs (if y
ou don't have one already
). \\

There's not a "standard
specification template" for all projects in all industries because the individual requirements
that populate an SRS are unique not only
 from company
 to company
, but also from project to
project within any
 one company.\\

The key
 is to select an existing template or specification to
begin with, and then adapt it to meet y
our needs.
In recommending using existing templates, I'm not advocating simply
 copy
ing a template
from available resources and using them as y
our own; instead, I'm suggesting that y
ou use
available templates as guides for developing y
our own. \\

It would be almost impossible to find
a specification or specification template that meets y
our particular project requirements
exactly
. But using other templates as guides is how it's recommended in the literature on
specification development. Look at what someone else has done, and modify
 it to fit your project requirements. (See the sidebar called "Resources for Model Templates" at the end of
this article for resources that provide sample templates and related information.)

\subsection*{A sample of a more detailed SRS outline}
\begin{enumerate}
\item Scope  
	\begin{enumerate}
	\item A Webinterface to CGS,
		make, Doxygen, HTML, CSS, PHP,
		Shell script, PNG
		images.
		\item
		\begin{enumerate}
		\item  Giving input to the interface in form of textfields. \\
		\item  storing the information in form of PHP variables to be processed for certificates. \\
		\item Processing the database
		based on the user queries and
		can send the output in the form of odt or pdf formats.
		 		\end{enumerate}
		\item  Document overview.
		The institution can use this
		project to infer some very
		 useful
		information in matter of seconds
		from the already
		 existent functions. At the same
		time the information is being
		stored in a database that can be
		made available to a community for recognition and reward authority.
		This document comprises six
		sections:
		\begin{itemize}
		\item Scope
		\item Referenced documents
		\item Requirements
		\item Testing and Coding 
				\end{itemize}
\end{enumerate}


\item Referenced Documents 
\begin{enumerate}
\item CGS technical
manual.
\item Material database.
\end{enumerate} 
\item Requirements 
\begin{enumerate}
\item An interface for easy operation.
\item A web interface for opearting commands and excessing the processed
data from any
 system over the
web.
\item Flexibility
 to incorporate any
changes in the standard sy
ntax of
the scripting
methods. 

\end{enumerate}
\end{enumerate}


